\documentclass{article}
\usepackage[left=1in,right=1in,top=1in,bottom=1in]{geometry}
\usepackage{listings}
\usepackage{graphicx}
\usepackage{hyperref}

\newcommand{\mytitle}{Cultural Mapping: AG 301}
\newcommand{\myauthor}{Alvin Johns}
\newcommand{\hsp}{\hspace{5mm}}

\title{\mytitle}
\author{\myauthor}

\begin{document}

\maketitle

%\includegraphics[width=1.0\textwidth]{pathtoimage.png}
%\caption{ImageDescription}

\center\section*{\underline{Historic Self}}\vspace{3mm}
\begin{minipage}[t]{1.0\linewidth}
    \begin{minipage}[t]{0.45\linewidth}
        \textbf{Who were your ancestors?}\vspace{1mm}\\
        My mother's tribe was Blackfoot and my father can trace his roots to Cape
        Verde, a group of islands nested off the north-west coast of Africa. My grandmother's parents arrived to the US in the early 1900's and 
        spoke fluent Portuguese. While my grandfather's roots are untracable, it is 
        safe to assume that his history can be traced back to slavery.\\

        \textbf{What languages did they speak?}\vspace{1mm}\\
        Portuguese + English.\\

        \textbf{What land-base did they come from?}\vspace{1mm}\\
        Cape Verdean Islands.\\

        \textbf{Did they leave? By choice?}\vspace{1mm}\\
        Yes, my grandmother's parents came here by choice. It is assumed that
        the family history of my grandfather arrived here by slavery.\\
    
        My mother's parents were already here, living in Montana.\\
    \end{minipage}\hsp
    \begin{minipage}[t]{0.45\linewidth}
        \textbf{What is your generational history?}\vspace{1mm}\\
        I can relate part of my family back to the 1900's. While I do know my 
        mother's history stems back before the US was formed, that history is 
        mostly unknown.\\

        \textbf{Are there gaps and do you know why?}\vspace{1mm}\\
        Yes there are gaps, one stemming from slavery, most likely, and the other
        reason is just a causation of life.\\

        \textbf{Was there migration and do you know why?}\vspace{1mm}\\
        My grandmother's parents came to the US during the early 1900's from
        Cape Verde.\\

        \textbf{How many generations can you relate and why?}\vspace{1mm}\\
        I have met my grandmother's brothers and sisters, visiting them on the
        east coast during my early years. My great-uncle from the
        Massachusetts police department and a great-aunt took care of the home
        and held family values in-tact. My most fond memories come from spending
        a great deal of time with another great-uncle, whom gave me the nickname:
        'shadow'. Cape Verdean culture was preserved in the family I have on the
        east coast; every year we would go back east to a family gathering, full
        of dancing and food.\\
    \end{minipage}
\end{minipage}

\center\section*{\underline{Contemporary Self}}\vspace{3mm}
\begin{minipage}[t]{1.0\linewidth}
    \begin{minipage}[t]{0.45\linewidth}
        \textbf{What languages survived?}\vspace{1mm}\\
        Portuguese.\\

        \textbf{What traditions survived?}\vspace{1mm}\\
        No significant passing of traditions.\\

        \textbf{What cultural ways survived?}\vspace{1mm}\\
        Food is a major part of my life, especially how food is cooked. Jagacita
        is always a welcomed dish and keeps me connected to the family I know.\\
    \end{minipage}\hsp
    \begin{minipage}[t]{0.45\linewidth}
        \textbf{Is your contemporary world view similar to that of your
        ancestors?}\vspace{1mm}\\
        I would say that my world views are mainly consistent with what
        generations before me believed. Maintaining essential values while 
        adapting and overcoming to the reality of any situation are primary 
        factors needed for any new generation. My ancestors would probably agree.  

        \textbf{Have you benefitted from previous generations?
        How?}\vspace{1mm}\\
        Of course. Each generation tries to improve the next. The very fact that
        I am here demonstrates that.

        \textbf{Have your contemporary values been influenced by previous familial
        generations? How?}\vspace{1mm}\\
        Yes they have, with the reasons being consistent with my contemporary
        world view. Certain issues facing current times are a product of having
        less pertanent tasks in their life. We were taught to sidestep if needed,
        bypassing what is not productive or mainly superficial.
    \end{minipage}
\end{minipage}

\end{document}
