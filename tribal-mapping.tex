\documentclass{article}
\usepackage{enumitem}

\title{Tribal Mapping Assignment}
\author{}
\date{\today}

\begin{document}

\maketitle

\section{Introduction}
The purpose of this assignment is to research, identify, and recognize tribes that live near you. This document outlines a structured approach to complete the Tribal Mapping assignment.

\section{Step 1: Locate at least Four Tribes}
Identify at least four tribes or indigenous groups from your local state or region where you have lived most of your life before attending OSU. For each tribe, gather the necessary information regarding their recognition status and historical context.

\section{Step 2: Identify Recognition Status}
Categorize each tribe as follows:
\begin{itemize}
    \item Federally Recognized
    \item State Recognized
    \item Non-Recognized
    \item Absorbed into a Larger Group
\end{itemize}

\section{Step 3: Write the Paper}
For each tribe, write a paragraph including the following details:
\begin{itemize}
    \item \textbf{Name of the Tribe}: Provide the full name of the tribe.
    \item \textbf{Location}: Describe the tribe's current location and its proximity to your home. Explain how you found this information. Ensure at least one tribe is from your home state/territory.
    \item \textbf{Original vs. Current Landscape}: Mention if the tribe is on its original land or if they have been moved. Include details of their original lands and resources available there. Include dates of tribal removals if applicable.
    \item \textbf{Interesting Facts}: Share interesting facts about each tribe. Make a connection to a concept from your class (landscapes, ecology, foods, forest management, basketry, etc.).
    \item \textbf{Visuals}: Include pictures if possible (optional).
\end{itemize}

\section{Example Outline for One Tribe}

\subsection{Tribe Name: The Cherokee Nation}
\begin{itemize}
    \item \textbf{Location}: The Cherokee Nation is primarily located in Oklahoma, but historically, they lived in the southeastern United States, including Georgia, North Carolina, and Tennessee. This information was obtained from the Cherokee Nation's official website and historical records.
    \item \textbf{Original vs. Current Landscape}: The Cherokee were removed from their original lands in the 1830s during the Trail of Tears, which relocated them to present-day Oklahoma. Their original lands were rich in resources like fertile soil, diverse wildlife, and ample water sources.
    \item \textbf{Interesting Facts}: The Cherokee developed a written syllabary, created by Sequoyah, which helped preserve their language. This connects to the class concept of cultural preservation and the role of language in maintaining cultural heritage.
\end{itemize}

\section{Resources for Research}
\begin{itemize}
    \item Tribal websites and official documents
    \item Local libraries and historical societies
    \item Government resources (e.g., Bureau of Indian Affairs)
    \item Academic articles and books on indigenous studies
\end{itemize}

\section{Final Steps}
\begin{itemize}
    \item Compile the information into a cohesive paper, ensuring clarity and proper structure.
    \item Proofread for accuracy and coherence.
    \item Add pictures if available and relevant.
\end{itemize}

\end{document}
