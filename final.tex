\documentclass{article}
\usepackage{amsmath}
\usepackage{amssymb}
\usepackage{graphicx}
\usepackage{float}
\usepackage{hyperref}
\usepackage{subcaption}
\usepackage{listings}
\usepackage{color}
\hypersetup{
    colorlinks=true,
    linkcolor=black,
    filecolor=black,      
    urlcolor=black,
}

\lstset{
  basicstyle=\ttfamily,
  columns=fullflexible,
  frame=single,
  breaklines=true,
}

\title{AG301 Final}
\author{Alvin Johns}
\date{02 June 2024}

\begin{document}

\maketitle

\raggedright

\section*{Introduction}

Geographic Information Systems (GIS) are powerful tools for analyzing and visualizing spatial data. Combining existing land management and burning practices with GIS can leverage these tools to restore and protect ecosystems on a greater scale.\vspace{5mm}

\section*{NASA's Fire Information for Resource Management System (FIRMS)}
Provides a near real-time view of global fire activity. FIRMS can be used to monitor active fires, track fire history, and assess fire damage. FIRMS data can be used to identify areas at risk of wildfires and inform land management decisions.\vspace{5mm}

Dynamic fire maps can be created to monitor fire behavior and aid in long-term planning. By integrating FIRMS data with GIS, land managers can identify areas at risk of wildfires and implement preventative measures.\vspace{5mm}

\section*{NASA's Earthdata}
The GIS toolset consists of a variety of components that researchers and fire managers visualize data through ineractive maps. The addtional insight provided by the GIS toolset can help land managers make informed decisions about fire management and restoration. \vspace{5mm}

When paired with burning practices known to restore ecosystems, GIS can be enhanced to better identify areas in need of restoration and develop a plan to implement controlled burns. \vspace{5mm}

Glenn Jones highlighted the difficulties of approving burns on small plots, citing the amount of beauracracy involved. The hope is that further research could not only improve the process of approving burns, but also provide a more comprehensive understanding of the benefits of controlled burns to the surrounding communities.\vspace{5mm}

\section*{Curriculum Incorporation}

A capstone or graduate project using GIS and FIRMS data could be developed to analyze fire activity in a specific region and develop a plan for restoration and fire management. Students could use GIS to analyze fire data, identify areas at risk of wildfires, and develop a plan for restoration and fire management.\vspace{5mm}

\section*{Conclusion}

Using, and improving upon, the research performed in Russia, the NASA data, and the GIS tools available, students can collaborate with land managers to approach prescribed burns in a more efficient and effective manner. Using simulations and predictive models, multiple plans can be developed for restoration and fire management that minimizes the risk of wildfires and reduce concerns of the local community.\vspace{5mm}

\newpage

\section*{Sources:}

\small
\begin{itemize}
    \sloppy
    \item NASA and USDA Forest Service. (2024). Fire Information for Resource Management System US / Canada. NASA.
    \url{https://firms.modaps.eosdis.nasa.gov/usfs/}
    \item NASA. (2024). GIS at NASA. NASA Earth Science.
    \url{https://www.earthdata.nasa.gov/learn/storymaps/gis-nasa} 
    \item Shadrin, D., Illarionova, S., Gubanov, F. et al. Wildfire spreading prediction using multimodal data and deep neural network approach. Sci Rep 14, 2606 (2024).
    \url{https://doi.org/10.1038/s41598-024-52821-x} 
\end{itemize}

\end{document}
